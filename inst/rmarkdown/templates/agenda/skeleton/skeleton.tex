\documentclass[]{article}
\usepackage{lmodern}
\usepackage{amssymb,amsmath}
\usepackage{ifxetex,ifluatex}
\usepackage{fixltx2e} % provides \textsubscript
\ifnum 0\ifxetex 1\fi\ifluatex 1\fi=0 % if pdftex
  \usepackage[T1]{fontenc}
  \usepackage[utf8]{inputenc}
\else % if luatex or xelatex
  \ifxetex
    \usepackage{mathspec}
  \else
    \usepackage{fontspec}
  \fi
  \defaultfontfeatures{Ligatures=TeX,Scale=MatchLowercase}
\fi
% use upquote if available, for straight quotes in verbatim environments
\IfFileExists{upquote.sty}{\usepackage{upquote}}{}
% use microtype if available
\IfFileExists{microtype.sty}{%
\usepackage[]{microtype}
\UseMicrotypeSet[protrusion]{basicmath} % disable protrusion for tt fonts
}{}
\PassOptionsToPackage{hyphens}{url} % url is loaded by hyperref
\usepackage[unicode=true]{hyperref}
\hypersetup{
            pdftitle={Meeting\_Name\_Here Agenda},
            pdfborder={0 0 0},
            breaklinks=true}
\urlstyle{same}  % don't use monospace font for urls
\usepackage[margin=2.54cm]{geometry}
\usepackage{longtable,booktabs}
% Fix footnotes in tables (requires footnote package)
\IfFileExists{footnote.sty}{\usepackage{footnote}\makesavenoteenv{long table}}{}
\IfFileExists{parskip.sty}{%
\usepackage{parskip}
}{% else
\setlength{\parindent}{0pt}
\setlength{\parskip}{6pt plus 2pt minus 1pt}
}
\setlength{\emergencystretch}{3em}  % prevent overfull lines
\providecommand{\tightlist}{%
  \setlength{\itemsep}{0pt}\setlength{\parskip}{0pt}}
\setcounter{secnumdepth}{0}
% Redefines (sub)paragraphs to behave more like sections
\ifx\paragraph\undefined\else
\let\oldparagraph\paragraph
\renewcommand{\paragraph}[1]{\oldparagraph{#1}\mbox{}}
\fi
\ifx\subparagraph\undefined\else
\let\oldsubparagraph\subparagraph
\renewcommand{\subparagraph}[1]{\oldsubparagraph{#1}\mbox{}}
\fi

% set default figure placement to htbp
\makeatletter
\def\fps@figure{htbp}
\makeatother


\title{Meeting\_Name\_Here Agenda}
\author{Chair: Chair\_name\_here\\
Attendees: Attendee\_One}
\date{Meeting to be held: Meeting\_date\_here, Meeting\_time\_here}

\begin{document}
\maketitle

\begin{longtable}[]{@{}ll@{}}
\toprule
\begin{minipage}[b]{0.16\columnwidth}\raggedright\strut
Who\strut
\end{minipage} & \begin{minipage}[b]{0.52\columnwidth}\raggedright\strut
Matters arising from previous minutes\strut
\end{minipage}\tabularnewline
\midrule
\endhead
\begin{minipage}[t]{0.16\columnwidth}\raggedright\strut
Name\strut
\end{minipage} & \begin{minipage}[t]{0.52\columnwidth}\raggedright\strut
Topic Topic line 2\strut
\end{minipage}\tabularnewline
\begin{minipage}[t]{0.16\columnwidth}\raggedright\strut
Name\strut
\end{minipage} & \begin{minipage}[t]{0.52\columnwidth}\raggedright\strut
Topic 2\strut
\end{minipage}\tabularnewline
\bottomrule
\end{longtable}

\begin{center}\rule{0.5\linewidth}{\linethickness}\end{center}

\begin{longtable}[]{@{}cl@{}}
\toprule
\begin{minipage}[b]{0.17\columnwidth}\centering\strut
Who\strut
\end{minipage} & \begin{minipage}[b]{0.20\columnwidth}\raggedright\strut
Topic\strut
\end{minipage}\tabularnewline
\midrule
\endhead
\begin{minipage}[t]{0.17\columnwidth}\centering\strut
Name\strut
\end{minipage} & \begin{minipage}[t]{0.20\columnwidth}\raggedright\strut
Topic \textbackslash{}n Topic line 2 is longer than line 1\strut
\end{minipage}\tabularnewline
\begin{minipage}[t]{0.17\columnwidth}\centering\strut
Name\strut
\end{minipage} & \begin{minipage}[t]{0.20\columnwidth}\raggedright\strut
Topic 2\strut
\end{minipage}\tabularnewline
\bottomrule
\end{longtable}

\begin{center}\rule{0.5\linewidth}{\linethickness}\end{center}

\begin{longtable}[]{@{}clrl@{}}
\caption{Here's the caption. It, too, may span multiple
lines.}\tabularnewline
\toprule
\begin{minipage}[b]{0.15\columnwidth}\centering\strut
Centered Header\strut
\end{minipage} & \begin{minipage}[b]{0.10\columnwidth}\raggedright\strut
Default Aligned\strut
\end{minipage} & \begin{minipage}[b]{0.20\columnwidth}\raggedleft\strut
Right Aligned\strut
\end{minipage} & \begin{minipage}[b]{0.31\columnwidth}\raggedright\strut
Left Aligned\strut
\end{minipage}\tabularnewline
\midrule
\endfirsthead
\toprule
\begin{minipage}[b]{0.15\columnwidth}\centering\strut
Centered Header\strut
\end{minipage} & \begin{minipage}[b]{0.10\columnwidth}\raggedright\strut
Default Aligned\strut
\end{minipage} & \begin{minipage}[b]{0.20\columnwidth}\raggedleft\strut
Right Aligned\strut
\end{minipage} & \begin{minipage}[b]{0.31\columnwidth}\raggedright\strut
Left Aligned\strut
\end{minipage}\tabularnewline
\midrule
\endhead
\begin{minipage}[t]{0.15\columnwidth}\centering\strut
First\strut
\end{minipage} & \begin{minipage}[t]{0.10\columnwidth}\raggedright\strut
row\strut
\end{minipage} & \begin{minipage}[t]{0.20\columnwidth}\raggedleft\strut
12.0\strut
\end{minipage} & \begin{minipage}[t]{0.31\columnwidth}\raggedright\strut
Example of a row that spans multiple lines.\strut
\end{minipage}\tabularnewline
\begin{minipage}[t]{0.15\columnwidth}\centering\strut
Second\strut
\end{minipage} & \begin{minipage}[t]{0.10\columnwidth}\raggedright\strut
row\strut
\end{minipage} & \begin{minipage}[t]{0.20\columnwidth}\raggedleft\strut
5.0\strut
\end{minipage} & \begin{minipage}[t]{0.31\columnwidth}\raggedright\strut
Here's another one. Note the blank line between rows.\strut
\end{minipage}\tabularnewline
\bottomrule
\end{longtable}

\end{document}
